\chapter{Programming with Dependent Types}

The primary purpose of introducing dependent types into programming is to
greatly enhance the precision in using types to capture program invariants.
Generally speaking, dependent types are types dependent on values of
expressions. For instance, $\tbool$ is a type constructor in ATS that forms
a type $\tbool(b)$ when applied to a given boolean value $b$. As this type
can only be assigned to the boolean value $b$, it is often referred to as a
singleton type. Clearly, the meaning of $\tbool(b)$ depends on the boolean
value $b$.  Similarly, $\tint$ is a type constructor in ATS that forms a
type $\tint(i)$ when applied to a given integer $i$. This type is also a
singleton type as it can only be assigned to the integer value $i$.  Many
other examples of dependent types are to be introduced gradually when this
chapter unfolds.  In particular, a means for the programmer to declare
dependent datatype constructors is to be presented in
Section~\ref{section:dependent_datatypes}.

\begin{figure}
\[\begin{array}{llcl}
\sim & (\verb`~`) & : & (\sint) \simp \sint \\
+ & (\verb`+`) & : & (\sint, \sint) \simp \sint \\
- & (\verb`-`) & : & (\sint, \sint) \simp \sint \\
\times & (\verb`*`) & : & (\sint, \sint) \simp \sint \\
\div & (\verb`/`) & : & (\sint, \sint) \simp \sint \\
> & (\verb`>`) & : & (\sint, \sint) \simp \sbool \\
\geq & (\verb`>=`) & : & (\sint, \sint) \simp \sbool \\
< & (\verb`<`) & : & (\sint, \sint) \simp \sbool \\
\leq & (\verb`<=`) & : & (\sint, \sint) \simp \sbool \\
= & (\verb`==`) & : & (\sint, \sint) \simp \sbool \\
\neq & (\verb`<>`) & : & (\sint, \sint) \simp \sbool \\
\lor & (\verb`||`) & : & (\sbool, \sbool) \simp \sbool \\
\land & (\verb`&&`) & : & (\sbool, \sbool) \simp \sbool \\
\end{array}\]
\caption{Some commonly used statics constants and their sorts}
\label{figure:static_constants}
\end{figure}
\section{Statics}
The statics of ATS is a simply typed language. The types and terms in this
language are referred to as sorts and static terms, respectively. Some of
the base sorts are given as follows:
\begin{itemize}
\item The sort $\sbool$ is for static terms of boolean values.
\item The sort $\sint$ is for static terms of integer values.
\item The sort $\stype$ is for static terms representing types of size
equal to one word.
\item The sort $\staype$ is for static terms representing types of
unspecified size.
\end{itemize}
There are predicative and impredicative sorts:
$\sbool$ and $\sint$ are predicative while $\stype$ and $\staype$ are
impredicative. If a static term is assigned a predicative sort, then
then the term is often referred to as a type index.

There are various constants in the statics.  In
Figure~\ref{figure:static_constants}, some commonly used static constant
functions are listed together with their sorts.  The symbols given inside
parentheses are the names that refer to these constants in the concrete
syntax. Many more static constants can be found in the following file:
\begin{center}
{\it\ATSHOME/prelude/basic\_sta.sats}.
\end{center}
The names for static constant functions can be overloaded, and an
overloaded name is resolved based on the arity information as well as the
information on the sorts of the arguments.

A subset sort is a sort restricted by a predicate. For instance, $\snat$ is
a subset sort defined as $\{a:\sint\mid a\geq 0\}$. In the concrete syntax,
this is done as follows:
\begin{verbatim}
sortdef nat = {a: int | a >= 0 }
\end{verbatim}
where {\it sortdef} is a keyword in ATS for introducing a sort definition.
It is important to not confuse sorts with subsets. The latter can only used
to classify quantified static variables.  For instance, the following type
(written in the concrete syntax)
can be assigned to a function that tests whether two given natural numbers
are equal:
\begin{verbatim}
{i,j:nat} (int (i), int (j)) -> bool (i == j)
\end{verbatim}
This type is essentially treated like syntactic sugar for the following
one (also written in the concrete syntax):
\begin{verbatim}
{i,j:int | i >= 0; j >= 0} (int (i), int (j)) -> bool (i == j)
\end{verbatim}

\begin{figure}[thp]
\[\begin{array}{rcl}
%{\it badd} & : & \forall b_1:\sbool.\forall b_2:\sbool.~(\tbool(b_1), \tbool(b_2))\timp\tbool(b_1\lor b_2) \\
%{\it bmul} & : & \forall b_1:\sbool.\forall b_2:\sbool.~(\tbool(b_1), \tbool(b_2))\timp\tbool(b_1\land b_2) \\
{\it ineg} & : & \forall i:\sint.~(\tint(i))\timp\tint(\sim\!i) \\
{\it iadd} & : & \forall i_1:\sint.\forall i_2:\sint.~(\tint(i_1), \tint(i_2))\timp\tint(i_1+i_2) \\
{\it isub} & : & \forall i_1:\sint.\forall i_2:\sint.~(\tint(i_1), \tint(i_2))\timp\tint(i_1-i_2) \\
{\it imul} & : & \forall i_1:\sint.\forall i_2:\sint.~(\tint(i_1), \tint(i_2))\timp\tint(i_1\times i_2) \\
{\it idiv} & : & \forall i_1:\sint.\forall i_2:\sint.~(\tint(i_1), \tint(i_2))\timp\tint(i_1\div i_2) \\
{\it igt} & : & \forall i_1:\sint.\forall i_2:\sint.~(\tint(i_1), \tint(i_2))\timp\tbool(i_1>i_2) \\
{\it igte} & : & \forall i_1:\sint.\forall i_2:\sint.~(\tint(i_1), \tint(i_2))\timp\tbool(i_1\geq i_2) \\
{\it ilt} & : & \forall i_1:\sint.\forall i_2:\sint.~(\tint(i_1), \tint(i_2))\timp\tbool(i_1<i_2) \\
{\it ilte} & : & \forall i_1:\sint.\forall i_2:\sint.~(\tint(i_1), \tint(i_2))\timp\tbool(i_1\leq i_2) \\
{\it ieq} & : & \forall i_1:\sint.\forall i_2:\sint.~(\tint(i_1), \tint(i_2))\timp\tbool(i_1=i_2) \\
{\it ineq} & : & \forall i_1:\sint.\forall i_2:\sint.~(\tint(i_1), \tint(i_2))\timp\tbool(i_1\not=i_2) \\
\end{array}\]
\caption{Some common arithmetic and comparison functions and their
dependent types}
\label{figure:arithmetic_comparison_functions}
\end{figure}
\section{Common Arithmetic and Comparison Functions}
Some commonly used arithmetic and comparison functions are listed in
Figure~\ref{figure:arithmetic_comparison_functions} together with the
dependent types assigned to them. In practice, overloaded names are often
used to refer to these functions. For instance, in the following function
definition, {\it\verb`>`} and {\it \verb`-`} are resolved into {\it igt}
and {\it isub}, respectively:
\begin{verbatim}
// [mul_int_int]: the integer mulplication function
fun fact {n:nat} (x: int n): int =
  if x > 0 then mul_int_int (x, fact (x - 1)) else 1
\end{verbatim}
Note that the symbol {\it int} is overloaded: it may represent a dependent
type constructor (as in $\tint(n)$) or just a type (for integers).  Many
more common functions on integers can be found in the following file:
\begin{center}
{\it\ATSHOME/prelude/SATS/integer.sats}.
\end{center}

\section{Constraint Solving}
Typechecking in ATS involves generating and solving constraints.  As an
example, the code below gives an implementation of the factorial function:
\begin{verbatim}
// this definition does not typecheck due to a nonlinear constraint
fun fact {n:nat}
  (x: int n): [r:nat] int r = if x > 0 then x * fact (x - 1) else 1
\end{verbatim}
In this implementation, the function ${\it fact}$ is assigned the
following type:
\[\begin{array}{l}
\forall n:\snat.~\tint(n)\timp\exists r:nat.~\tint(r)
\end{array}\]
which means that ${\it fact}$ returns a natural number $r$ when applied to
a natural number $n$. When the code is typechecked, the following
constraints need to be solved:
\[\begin{array}{ll}
1. & \forall n:nat.~n > 0 \limplies n - 1 \geq 0 \\
2. & \forall n:nat.\forall r_1:nat.~n > 0 \limplies n * r_1 \geq 0 \\
3. & \forall n:nat.~n > 0 \limplies 1 \geq 0 \\
\end{array}\]
The first constraint is generated due to the call ${\it fact} (x-1)$, which
requires that $x-1$ be a natural number. The second constraint is generated
in order to verify that $x * {\it fact} (x-1)$ is a natural number under
the assumption that ${\it fact} (x-1)$ is a natural number.  The third
constraint is generated in order to verify that $1$ is a natural number.
The first and the third constraints can be readily solved by the constraint
solver in ATS, which is based on the Fourier-Motzkin variable elimination
method~\cite{DE73}. However, the second constraint cannot be handled by the
constraint solver as it is nonlinear: The constraint cannot be turned into
a linear integer programming problem due to the occurrence of the nonlinear
term $n*r_1$.

While nonlinear constraints cannot be handled automatically by the
constraint solver in ATS, the programmer is given a means to verify them by
constructing proofs in ATS explicitly. The issue of explicit proof
construction is to be elaborated elsewhere.

\section{A Simple Example: Dependent Types for Debugging}
Given a non-negative integer $x$, the integer square root of $x$ is the
greatest integer $i$ satisfying $i * i \leq x$.  In
Figure~\ref{figure:isqrt_function_nondependent}, an implementation of the
integer square root function is given based on the method of binary search.
This implementation passes typechecking, but it seems to be looping forever
when tested. Instead of going into the standard routine of debugging (e.g.,
by inserting calls to some printing functions), we now attempt to identify
the cause for non-termination by trying to prove the termination of the
function {\it search} through the use of dependent types.
\begin{figure}[thp]
\begin{verbatim}
fn isqrt (x: int): int = let
  fun search (x: int, l: int, r: int): int = let
    val diff = r - l
  in
    case+ 0 of
    | _ when diff > 0 => let
        val m = l + (diff / 2)
      in
        // [div_int_int] is the integer division function
        if div_int_int (x, m) < m then search (x, l, m) else search (x, m, r)
      end // end of [if]
    | _ => l
  end // end of [search]
in
  search (x, 0, x)
end // end of [isqrt]
\end{verbatim}
\caption{A buggy implementation of the integer square root function}
\label{figure:isqrt_function_nondependent}
\end{figure}
\begin{figure}[thp]
\begin{verbatim}
fn isqrt {x:nat} (x: int x): int = let
  fun search {x,l,r:nat | l < r} .<r-l>.
    (x: int x, l: int l, r: int r): int = let
    val diff = r - l
  in
    case+ 0 of
    | _ when diff > 1 => let
        val m = l + (diff / 2)
      in
        // [div_int_int] is the integer division function
        if div_int_int (x, m) < m then search (x, l, m) else search (x, m, r)
      end // end of [if]
    | _ => l
  end // end of [search]
in
  if x > 0 then search (x, 0, x) else 0
end // end of [isqrt]
\end{verbatim}
\caption{%
A fixed dependently typed implementation corresponding to the one
in Figure~\ref{figure:isqrt_function_nondependent}
}
\label{figure:isqrt_function_dependent}
\end{figure}

The function {\it search} in Figure~\ref{figure:isqrt_function_nondependent}
is assigned the type $(\tint, \tint, \tint) \timp \tint$, meaning that {\it
search} takes three integers as its arguments and returns an integer as its
result.  On the other hand, the programmer may have thought that the
function {\it search} should possess the following invariants (if
implemented correctly):
\begin{enumerate}
\item
$l*l\leq x < r*r$ must hold when ${\it search}(x, l, r)$ is called.
\item
Assume $l*l\leq x < r*r$ for some integers $x, l, r$.  If a recursive call
${\it search}(x, l', r')$ for some integers $l'$ and $r'$ is encountered in
the body of ${\it search}(x, l, r)$, then $r'-l' < r - l$ must hold. This
invariant implies that {\it search} is terminating.
\end{enumerate}
Though the first invariant can be captured in the type system of ATS, it is
somewhat involved to do so due to the need for handling nonlinear
constraints. Instead, the following dependent function type is assigned to
${\it search}$ in Figure~\ref{figure:isqrt_function_dependent}, which
captures a weaker invariant stating that $l < r$ must hold when ${\it
search}(x, l, r)$ is called:
\[\begin{array}{lcr}
{\it search} & : & \forall x:\snat.\forall l:\snat.\forall r:\snat.
(l < r) \Bimp \langle r-l\rangle \metimp ((\tint(x), \tint(l), \tint(r))\timp\tint)
\end{array}\]
It should not be difficult to relate this type to the concrete syntax
representing it in the code. Note that the term $\langle r-l\rangle$, which
corresponds to the concrete syntax \verb`.<r-l>.`, represents a termination
metric needed for verifying the second invariant.  More details on
termination metrics are to be given later.

With {\it search} being assigned the above dependent function type, the
implementation in Figure~\ref{figure:isqrt_function_nondependent} needs to
be modified in order to pass typechecking. The modifications can be readily
identified by comparing the code in
Figure~\ref{figure:isqrt_function_nondependent} to the the code in
Figure~\ref{figure:isqrt_function_dependent}, and the reader is strongly
encouraged to do so and then figure out the reason for these
modifications.

\section{Dependent Datatypes}\label{section:dependent_datatypes}
The feature of datatypes in ATS is directly adopted from ML. In ATS, there
is even a means for the programmer to introduce dependent datatypes (or
more precisely, dependent datatype constructors). For instance, the
datatype constructor $\tlist$ in ATS is declared as follows:
\begin{verbatim}
datatype list (t@ype+, int) =
  | {a:t@ype} list_nil (a, 0)
  | {a:t@ype} {n:nat} list_cons (a, n+1) of (a, list (a, n))
\end{verbatim}
The syntax indicates that $\tlist$ is a type constructor that forms a type
$\tlist(T,I)$ when applied to a type $T$ (of sort ${\it t@ype}$) and an
integer $I$. The sort $\staype$ means that the size of $T$ is
unspecified. The plus sign following ${\it t@ype}$ states that $\tlist$ is
covariant in its first argument, that is, $\tlist(T_1, n)$ is a subtype of
$\tlist(T_2, n)$ if $T_1$ is a subtype of $T_2$. There are two data
constructors ${\it list\_nil}$ and ${\it list\_cons}$ associated with
$\tlist$, which are assigned the following types:
\[\begin{array}{rcl}
{\it list\_nil} & : & \forall a:\staype.~() \timp \tlist (a, 0) \\
{\it list\_cons} & : & \forall a:\staype\forall n:\sint.~(a, \tlist(a, n)) \timp list (a, n+1) \\
\end{array}\]
Given a type $T$ and an integer $I$, it is clear that the type
$\tlist(T,I)$ is for lists of length $I$ in which each element is of type
$T$. The above declaration for $\tlist$ can also be given as follows in a
more succint manner:
\begin{verbatim}
datatype list (a:t@ype+, int) =
  | list_nil (a, 0) | {n:nat} list_cons (a, n+1) of (a, list (a, n))
\end{verbatim}

\begin{figure}[thp]
\begin{verbatim}
// list_length<a>: {n:nat} list (a, n) -> int (n)
fn{a:t@ype} list_length {n:nat} (xs: list (a, n)): int (n) = let
  // loop: {i,j:nat} (list (a, i), int (j)) -> int (i+j)
  fun loop {i,j:nat} .<i>. // .<i>. is a termination metric
    (xs: list (a, i), j: int j): list (i+j) = begin
    case+ xs of list_cons (_, xs) => loop (xs, j+1) | list_nil () => j
  end // end of [loop]
in
  loop (xs, 0)
end // end of [list_length]
\end{verbatim}
\caption{A tail-recursive implementation of the list length function}
\label{figure:list_length_function}
\end{figure}
In Figure~\ref{figure:list_length_function}, an implementation of the list
length function is given. The function template {\it list\_length} can be
instantiated with a type $T$ (of unspecified size) to yield a function
${\it list\_length}\langle T\rangle$ of the following function type:
\[\begin{array}{l}
\forall n:\snat.~\tlist(T, n)\timp \tint(n) \\
\end{array}\]
which clearly indicates that the value returned by the function
${\it list\_length}\langle T\rangle$ is the length of its argument.

\section{Pattern Matching}
The feature of pattern matching in ATS is adopted from ML. However,
there are some interesting issues with pattern matching that occur
only in the presence of dependent datatypes~\cite{DTPM-jucs03}.

\subsection{Exhaustiveness}
A funcion template is implemented as follows:
\input{DATS/list_zip_with_function_1.dats}
Given two lists $v_{1,1},\ldots, v_{1,n}$ and $v_{2,1},\ldots, v_{2,n}$
where $v_{1,i}$ and $v_{2,i}$ are of types $T_1$ and $T_2$ for $1\leq i\leq
n$, the function call ${\it list\_zip\_with}\langle T_1,T_2\rangle$ is
expected to return a list $v_1,\ldots,v_n$ such that
$v_i=f(v_{1,i},v_{2,i})$ for $1\leq i\leq n$.  By the way, ${\it
list\_zip\_with}$ is also often referred to as ${\it list\_map2}$ in the
literature.  The use of the keyword $\KWcaseplus$ indicates that the
typechecker of ATS is able to verify the exhaustiveness of pattern matching
in this example. If ${\it xs1}$ matches a non-empty list while ${\it xs2}$
matches an empty one, the typechecker essentially generates an assumption
stating that $n=n_1+1$ and $n=0$, where $n_1$ is a newly introduced
variable ranging over natural numbers. As this is a contradictory
assumption, the case is ruled out. Similarly, the case where ${\it xs1}$
matches an empty list while ${\it xs2}$ matches a non-empty one is also
ruled out.  As there are no other cases, the exhaustiveness of pattern
matching is verified.

A slightly different implementation of ${\it list\_zip\_with}$ can be
done as follows:
\input{DATS/list_zip_with_function_2.dats}
In this implementation, the keyword $\KWvalplus$ indicates that the
pattern matching following it is exhaustive. Hence, the head and tail of
${\it xs2}$ can be extracted out without testing whether ${\it xs2}$ is
empty.

\subsection{Sequentiality}
In ATS, pattern matching is performed sequentially at run-time. In other
words, a clause is selected only if the given value matches the pattern
associated with this clause but they do not match the patterns associated
with the clauses ahead of it. Naturally, one might expect that the
following implementation of ${\it list\_zip\_with}$ also typechecks:
\input{DATS/list_zip_with_function_3.dats}
This, however, is not the case. In ATS, typechecking clauses is done
nondeterministically (rather than sequentially). In this example, the
second clause fails to typecheck as it is done without assuming that the
given value does not match the pattern associated with the first clause.
The second clause can be modified as follows:
\begin{verbatim}
  | (_, _) =>> list_nil ()
\end{verbatim}
The use of \verb`=>>` (in place of \verb`=>`) indicates to the typechecker
that this clause needs to be typechecked under the assumption that the
given value does not match the pattern associated with each previous
clause.  Hence, when the modified second clause is typechecked, it can be
assumed that the value that matches the pattern $(\_, \_)$ must match one
of the following:
$$\begin{array}{lll}
(list\_cons(\_,\_), list\_nil ())~~~& 
(list\_nil (), list\_cons(\_,\_))~~~&
(list\_cons(\_, \_), list\_cons(\_,\_)) \\
\end{array}$$
This assumption allows typechecking to succeed.

%%% end of \section{Pattern Matching}

\section{Program Termination Verification}
A termination metric is a tuple of natural numbers $\langle
m_1,\ldots,m_n\rangle$, and the standard lexicographic ordering on natural
numbers is used to order such tuples.  In ATS, termination metrics can be
supplied (by the programmer) for verifying whether recursively defined
functions are terminating, and this feature plays a crucial role in
supporting the paradigm of programming with theorem proving.

In the following example, a singleton metric $\langle n\rangle$ is supplied
to verify that the recursive function ${\it fact}$ is terminating, when $n$
is the value of the argument of ${\it fact}$:
\begin{verbatim}
fun fact {n:nat} .<n>. (x: int n): int = if x > 0 then x * fact (x-1) else 1
\end{verbatim}
The metric attached to the call ${\it fact}(x-1)$ is $\langle n-1\rangle$,
which is obviously less than $\langle n\rangle$.

A more difficult and also more interesting example is given as follows,
where the MacCarthy's 91-function is implemented:
\input{DATS/f91_function.dats}
It is clear from the dependent type assigned to ${\it f91}$ that
the function always returns 91 when applied to an integer less than or
equal 100. The metric supplied to verify the termination of
${\it f91}$ is $\langle\max(101-i,0)\rangle$, where $i$ is the
value of the argument of ${\it f91}$.

The following code implements the Ackermann function, which is well-known
for being recursive but not primitive recursive:
\input{DATS/ackermann_function.dats}
The metric supplied for verifying the termination of ${\it ack}$ is
a pair $\langle m,n\rangle$, where $m$ and $n$ are the values of the
arguments of ${\it ack}$.

In the following example, {\it isEven} and {\it isOdd} are defined mutually
recursively:
\input{DATS/isEvenOdd_functions_termination.dats}
The metrics supplied for verifying the termination of ${\it isEven}$ and
${\it isOdd}$ are $\langle 2*n\rangle$ and $\langle 2*n+1\rangle$,
respectively, when $n$ is the value of the argument of ${\it isEven}$ and
also the value of the argument of ${\it isOdd}$.  Clearly, if the metrics
$\langle n, 0\rangle$ and $\langle n, 1\rangle$ are supplied for ${\it
isEven}$ and ${\it isOdd}$, respectively, the termination of these two
functions can also be verified.  Note that it is required that the metrics
for mutually recursively defined functions be of the same length.

%%% end of \chapter{Programming with Dependent Types} %%%
