This manual documents {\em ATS/Anairiats version x.x.x}, which is the
current released implementation of the programming language ATS. This
implementation itself is nearly all written in ATS.

The core of ATS is a call-by-value functional programming language equipped
with a type system rooted in the framework {\em Applied Type
System}~\cite{ATS-types03}. In particular, both dependent types and linear
types are supported in ATS. The dependent types in ATS are directly based
on those developed in Dependent ML (DML), an experimental programming
language that is designed in an attempt to extend ML with support for
practical programming with dependent types~\cite{DML-jfp07}. As of now, ATS
fully supersedes DML. While the notion of linear types is a familiar one in
programming lanugage research, the support for practical programming with
linear types in ATS is unique: It is based on a programming paradigm in
which programming is combined with theorem-proving.

The type system of ATS is stratified, consisting of a static component
(statics) and a dynamic component (dynamics). Types are formed and reasoned
about in the statics while programs are constructed and evaluated in the
dynamics. There is also a theorem-proving system ATS/LF built within ATS,
which plays an indispensable role in supporting the paradigm of programming
with theorem-proving. ATS/LF can also be employed to encode various
deduction systems and their meta-properties.

There is no support for program extraction (from proofs) in ATS. Instead,
the ATS compiler (Anairiats) erases all the types and proofs contained in a
program after it passes typechecking, and then translates the obtained
erasure into C code that can be further compiled into object code by a C
compiler such as GCC.

ATS programs can run with or without run-time garbage collection. The time
and space efficiency of the C code generated from a program written in ATS
often rivals that of its counterpart written in C (or \cplusplus)
directly. This is of particular importance when ATS is used for systems
programming.

%% The organization of the manual is given as follows:
%% \begin{itemize}
%% \item{\bf Part I}, {\em An Introduction to ATS}, gives an overview of the
%% programming language ATS.
%% \end{itemize}
