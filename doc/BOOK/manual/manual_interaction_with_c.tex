\chapter{Interaction with C}\label{chapter:interaction_with_c} As ATS and C
share precisely the same data representation, interaction between ATS and C
is mostly done in a straightforward manner.  However, it should be
emphasized that type safety can be compromised due to such interaction,
and thus it is suggested that this be done with great caution.

\begin{figure}[thp]
\begin{verbatim}
extern fun fact (x: int): int = "fact_extern"

%{^

/* external C code to be put at the top */

ats_int_type fact_extern (ats_int_type x) {
  int i, res ;
  res = 1 ; for (i = 1; i <= x; i += 1) res *= i ;  
  return res ;
} /* end of [fact_extern] */

%}

implement main () = begin
  print "fact (10) = "; print (fact 10); print_newline ()
end
\end{verbatim}
\caption{A simple example involving external C code}
\label{figure:fact_extern_function_1}
\end{figure}

\begin{figure}[thp]
\begin{verbatim}
extern fun fact (x: int): int = "fact_extern"

implement fact (x) = if x > 0 then x * fact (x - 1) else 1

%{$

/* external C code to be put at the bottom */

ats_void_type mainats () {
  printf ("fact (10) = %i\n", fact_extern (10)) ; return ;
}

%}

implement main_dummy () = () // [mainats] is implemented in C
\end{verbatim}
\caption{Another simple example involving external C code}
\label{figure:fact_extern_function_2}
\end{figure}
\section{External C Code}
A function declaration may attach an external name to the declared
function, allowing it to be referred to outside ATS. In
Figure~\ref{figure:fact_extern_function_1}, ${\it fact}$ is declared to
be a function from integers to integers.  This function is given an
external name ${\it fact\_extern}$.  In ATS, extern C code is allowed to
appear inside the following special pairs of parentheses:
\begin{itemize}
\item \verb`%{` and \verb`}%`: The C code enclosed by this pair is to be
relocated to somewhere (unspecified) in the code generated from compiling
the file containing the C code.
\item \verb`%{^` and \verb`}%`:
The C code enclosed by this pair is to be relocated to the top of the
code generated from compiling the file containing the C code.
\item \verb`%{$` and \verb`}%`:
The C code enclosed by this pair is to be relocated to the bottom of the
code generated from compiling the file containing the C code.
\end{itemize}
In Figure~\ref{figure:fact_extern_function_1}, a function of the name {\it
fact\_extern} is implemented in C. Note that the type ${\it
ats\_int\_type}$ in C is the counterpart of the type $\tint$ in ATS.  When
the code in Figure~\ref{figure:fact_extern_function_1} is compiled, the
call to ${\it fact}$ (on the integer 10) in ATS is translated to a call to
${\it fact\_extern}$. It may be helpful if the reader compiles this example
and then takes a look at the emitted C code.

In Figure~\ref{figure:fact_extern_function_2}, the function {\it fact} is
implemented in ATS. When compiled, this implementation is translated into
an implementation of ${\it fact\_extern}$ in C.  The function {\it main} in
ATS is given the external name {\it mainats}.  In
Figure~\ref{figure:fact_extern_function_2}, a function of this name is
implemented in C, where a call to {\it fact\_extern} is made. Note that the
type ${\it ats\_void\_type}$ in C is the counterpart of the type ${\it
void}$ in ATS.  Also, a function ${\it main\_dummy}$ is implemented in
Figure~\ref{figure:fact_extern_function_2}.  The sole purpose for this
implementation is to indicate to the ATS compiler (atsopt) that ${\it
mainats}$ is implemented externally.

\begin{figure}
\input{external_list0_length_function.dats}
\caption{An implementation of the list length function in C}
\label{figure:external_list0_length_function}
\end{figure}
The code in Figure~\ref{figure:external_list0_length_function} gives
another typical use of external C code. In this example, the functions
${\it list0\_is\_nil}$ and ${\it list0\_tail}$ are both implemented in
ATS, but the function ${\it list0\_length}$ is implemented in C.

\section{External Types}
Suppose that the name {\it someType} refers to some type declared in C.  Then
this type can be referred to as {\it \$extype "someType"} in ATS. On the
other hand, one can introduce external names for types in ATS and then use
such names outside ATS. For instance, an external name {\it
int\_int\_pair} is introduced in the following code to refer to the type
{\it @(int, int)}:
\begin{verbatim}
extern typedef "int_int_pair" = @(int, int)
\end{verbatim}
In this case, ${\it int\_int\_pair}$ is essentially bound to a struct type
in C as follows:
\begin{verbatim}
typedef struct {
  ats_int_type atslab_0 ; ats_int_type atslab_1 ;
} int_int_pair ;
\end{verbatim}
Note that \verb`atslab_` is the prefix used by the ATS compiler to form
labels for field selection.

\begin{figure}
\input{external_list0_append_function.dats}
\caption{An implementation of the list append function in C}
\label{figure:external_list0_append_function}
\end{figure}
Suppose that $v$ is a value of the form $C(v_1,\ldots, v_n)$, where $C$ is
a constructor associated with some datatype and $v_1,\ldots,v_n$ are values
of types $T_1,\ldots, T_n$, respectively. The value $v$ is represented by a
pointer to some struct when compiled into C, and the type of this pointer
can be referred to as $C\_{\it pstruct}(T_1,\ldots, T_n)$ in ATS. As an
example, the funtion for appending two lists together is implemented
externally in Figure~\ref{figure:external_list0_append_function}. The
reader may want to compile this example and then carefully inspect the
emitted C code.

\section{External Values}
Suppose that the name {\it someValue} refers to some value in C. Then this
value can be referred to as {\it \$extval ($T$, "someValue")}, where $T$ is
the perceived type of this value in ATS. For instance, {\it stdin\_ref} is
defined as a macro in ATS:
\begin{verbatim}
macdef stdin_ref = $extval (FILEref, "stdin")
\end{verbatim}
where ${\it FILEref}$ is a type in ATS that approximately corresponds to
the type ${\it FILE*}$ in C. On the other hand, one can introduce external
names for values in ATS and then use such names outside ATS. For instance,
an external name {\it one\_one\_pair} is introduced in the following code
to refer to the value {\it @(1, 1)}:
\begin{verbatim}
extern val "one_one_pair" = @(1, 1)
\end{verbatim}
Note that each external value is registered as a global root for the
garbage collector in case garbagage collection is performed at run-time.

%%% end of [manual_interaction_with_c.tex] %%%
