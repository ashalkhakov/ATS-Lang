\chapter{Programming with Linear Types}

ATS is currently one of the few languages that have successfully made
linear types available for effective use in practice, and the paradigm of
programming with theorem proving plays an indispensable role in achieving
this.  In the rest of this chapter, we present some examples of resource
manipulation involving linear types. In particular, we demonstrate that ATS
not only supports flexible uses of pointers but also guarantees based on
its type system that such uses are safe.

\section{Flexible and Safe Uses of Pointers}
In ATS, a linear prop is referred to as a {\it view} and a linear type,
which is often a type combined with a view, is referred to as a {\it
viewtype}. A commonly used view constructor is $@$ (infix), which
forms a view $T@L$ when applied to a type $T$ and a memory location $L$. If
a proof of the view $T@L$ is available, then a value of the type $T$ must
be stored at the location $L$.

\begin{verbatim}
fn{a1,a2:t@ype | sizeof a1 == sizeof a2} swap {l1,l2:addr}
  (pf1: a1 @ l1 >> a2 @ l1, pf2: a2 @ l2 >> a1 @ l2 | p1: ptr l1, p2: ptr l2)
  : void = let
  val tmp = !p1
in
  !p1 := !p2; !p2 := tmp
end // end of [swap]
\end{verbatim}


%%% end of \chapter{Programming with Linear Types}
