\chapter{Programming with Linear Types}

The paradigm of programming with theorem proving plays an indispensable
role in making linear types available for practical use in ATS.  In this
chapter, we present some examples of resource manipulation involving linear
types. In particular, we demonstrate that ATS not only supports flexible
uses of pointers but can also guarantee based on its type system that such
uses are safe.

\section{Safe Memory Access through Pointers}
In ATS, a linear prop is referred to as a {\it view} and a linear type,
which is often a type combined with a view, is referred to as a {\it
viewtype}. A commonly used view constructor is $@$ (infix), which forms a
view $T@L$ when applied to a type $T$ and a memory location $L$. If a proof
of the view $T@L$ is available, then it is guaranteed that a value of the
type $T$ is stored at the location $L$. In the following presentation,
views of the form $T@L$ is often referred to as @-views.  As an example,
the following function templates {\it ptr\_get0} and {\it ptr\_set0}, which reads
and writes through a given pointer, are assigned types containing @-views:
\begin{verbatim}
fun{a:t@ype} ptr_get0 {l:addr} (pf: a @ l | p: ptr l): @(a @ l | a)
fun{a:t@ype} ptr_set0 {l:addr} (pf: a? @ l | p: ptr l, x: a): @(a @ l | void)
\end{verbatim}
Note that $\tptr$ is a type constructor that forms a type $\tptr(L)$ when
applied to a static term $L$ of the sort {\it addr}, and the only value of
the type $\tptr(L)$ is the pointer that points to the location represented
by $L$.

Given a type $T$, the function ${\it ptr\_get0}\langle
T\rangle$ is assigned the following type:
\[\forall l:\saddr.~(T @ l \mid \tptr (l))\timp(T@l \mid T)\]
This type indicates that the function ${\it ptr\_get0}\langle T\rangle$
returns a proof of the view $T@L$ and a value of the type $T$ when applied
to a proof of the view $T@L$ and a pointer of the type $\tptr(L)$ for some
$L$. Intuitively speaking, a proof of the view $T@L$, which is a form of
resource as $T@L$ is linear, is {\it consumed} when it is passed to ${\it
ptr\_get0}\langle T\rangle$, and another proof of the view $T@L$ is
generated when ${\it ptr\_get0}\langle T\rangle$ returns. Notice that a
proof of the view $T@L$ must be returned for otherwise subsequent accesses
to the memory location $L$ become impossible.

Similarly, the function ${\it ptr\_set0}\langle T\rangle$ is
assigned the following type:
\[\forall l:\saddr.~(T? @ l \mid \tptr (l))\timp(T@l \mid void)\]
Note that $T?$ is a type for values of size ${\it sizeof}(T)$ (that are
assumed to be uninitialized).  The function ${\it ptr\_set0}\langle
T\rangle$ returns a proof of the view $T@L$ when applied to a proof of the
view $T?@L$, a pointer of the type $\tptr(L)$ and a value of the type
$T$. The use of the view $T?@L$ indicates that the memory location at $L$
is assumed to be uninitialized when ${\it ptr\_set0}\langle T\rangle$ is
called.

\begin{figure}
\input{DATS/swap0_example.dats}
\caption{A function for swapping memory contents (I)}
\label{figure:swap0_example.dats}
\end{figure}
In Figure~\ref{figure:swap0_example.dats}, a function template {\it swap0}
is implemented for swapping memory contents at two given
locations. Compared to a corresponding implementation in C, the verbosity
of this one in ATS is evident. In particular, the need for {\it threading}
linear proofs through calls to functions that make use of resources often
results in a significant amount of code to be written. We now introduce
some special syntax to significantly alleviate the need for such code.

The function templates ${\it ptr\_get1}$ and ${\it ptr\_set1}$ are given
the following types:
\begin{verbatim}
fun{a:t@ype} ptr_get1 {l:addr} (pf: !a @ l >> a @ l | p: ptr l): a
fun{a:t@ype} ptr_set1 {l:addr} (pf: !a? @ l >> a @ l | p: ptr l, x: a): void
\end{verbatim}
Clearly, for each type $T$, the function
${\it ptr\_get1}\langle T\rangle$ is assigned the following type:
\[
\forall l:\saddr.~(!T@l \gg T@l \mid \tptr(l))\timp T
\]
Given a linear proof ${\it pf}$ of the view $T@L$ for some $L$ and a
pointer $p$ of the type $\tptr(L)$, the function call ${\it
ptr\_get1}\langle T\rangle({\it pf}, p)$ is expected to return a
value of the type $T$. However, the proof ${\it pf}$ is not
consumed. Instead, it is still a proof of the view $T@L$ after the function
call. Similarly, the function
${\it ptr\_set1}\langle T\rangle$ is assigned the following type:
\[
\forall l:\saddr.~(!T?@l \gg T@l \mid \tptr(l), T)\timp\tvoid
\]
Given a linear proof ${\it pf}$ of the view $T?@L$ for some $L$, a pointer
$p$ of the type $\tptr(L)$ and a value $x$ of the type $T$, the function
call ${\it ptr\_set1}\langle T\rangle({\it pf}, p, x)$ is expected to
return the void value while changing the view of ${\it pf}$ from $T?@L$ to
$T@L$.  In general, if $f$ is given a type of the following form for some
views $V_1$ and $V_2$: $$(\ldots,!V_1\gg V_2,\ldots)\timp\ldots$$ then a
function call $f(\ldots,{\it pf},\ldots)$ on some proof variable ${\it pf}$
of the view $V_1$ is to change the view of ${\it pf}$ into $V_2$ upon its
return.  In the case where $V_1$ and $V_2$ are the same, $!V_1\gg V_2$ can
simply be written as $!V_1$.
\begin{figure}
\input{DATS/swap1_example.dats}
\caption{A function for swapping memory contents (II)}
\label{figure:swap1_example.dats}
\end{figure}
As an example, a function {\it swap1} for swapping the contents at two
given memory locations is implemented in
Figure~\ref{figure:swap1_example.dats}, where the function templates ${\it
ptr\_get1}$ and ${\it ptr\_set1}$ are employed.  Clearly, this
implementation is considerably cleaner when compared to the one in
Figure~\ref{figure:swap0_example.dats}.

\begin{figure}
\input{DATS/swap1_alt_example.dats}
\caption{A function for swapping memory contents (III)}
\label{figure:swap1_alt_example.dats}
\end{figure}
A further simplied implementation of {\it swap1} is given in
Figure~\ref{figure:swap1_alt_example.dats}. Given a pointer $p$ of the type
$\tptr(L)$ for some $L$, $!p$ yields the value stored at the memory
location $L$.  The typechecker first searches for a proof of the view $T@L$
for some $T$ among all available proofs when typechecking $!p$; if such a
proof ${\it pf}$ is found, then $!p$ is essentially elaborated into ${\it
ptr\_get1}({\it pf} \mid p)$ and then typechecked. As $!p$ is a left-value,
it can also be used to form an assignment like $!p := v$. The typechecker
elaborates $!p := v$ into ${\it ptr\_set1}({\it pf} \mid p, v)$ for the
sake of typechecking if a proof of the view $T@L$ can be found among all
available proofs.

\section{Local Variables}
Local variables within a function are stored in the frame allocated for the
function when it is called. In ATS, it is guaranteed by the type system
that no local variables stored in the frame of a function can be accessed
once the call to the function returns. In other words, the issue of a local
variable escaping its legal scope is completely prevented by the type
system of ATS.

\begin{figure}[thp]
\input{DATS/fact_function_var.dats}
\caption{An implementation of the factorial function that makes use of a local variable}
\label{figure:fact_function_var.dats}
\end{figure}
In Figure~\ref{figure:fact_function_var.dats}, a local variable is used in
an implementation of the factorial function. Given an address $L$, the
function {\it loop} in the implementation takes a proof of $\tint@L$, an
integer and a pointer to $L$, and it keeps updating the integer value
stored at $L$ until it exits. When the local variable ${\it res}$ is
introduced, a static address of the same name is introduced implicitly. In
addition, a proof of the view $\tint?@{\it res}$ is introduced implicitly,
and this proof is often referred to as the proof assocatiated with the
dynamic variable ${\it res}$.  In order to guarantee that ${\it res}$ can
never be accessed outside its legal scope, the type system requires that at
the end of the legal scope of ${\it res}$, the view of the proof associated
with ${\it res}$ must be the same as the view originally assigned to the
proof.

The expression ${\it \&res}$, which is assigned the type $\tptr(res)$,
represents the pointer to ${\it res}$, and the expression ${\it view@}({\it
res})$ refers to the implicitly introduced proof that is associated with
${\it res}$. Note that the view of this proof changes from $\tint?@{\it
res}$ into $\tint@{\it res}$ once ${\it res}$ is initialized with the
integer $1$.

If the following line is used to introduce ${\it res}$
instead:
\begin{verbatim}
  var res: int // uninitialized
\end{verbatim}
then the proof associated with ${\it res}$ is assigned the view
$\tint?@{\it res}$ when it is passed as a proof argument to the function
{\it loop}, resulting in a type error.

It is also possible to employ the following syntax to introduce the local
variable ${\it res}$ in Figure~\ref{figure:fact_function_var.dats}:
\begin{verbatim}
  var res: int with pf_res = 1
\end{verbatim}
In addition to introducing ${\it res}$, the form of syntax also introduces
${\it pf\_res}$ as an alias of ${\it view@}({\it res})$, thus allowing the
former to be used in place of the latter.

The implementation of the factorial function in
Figure~\ref{figure:fact_function_var.dats} should be compared with the one
in Figure~\ref{figure:fact_function_ref.dats}, where a reference (instead
of a local variable) is created to hold intermediate results during
computation. As references are allocated on heap and the memory for storing
each reference can only be safely reclaimed through GC, using local
variables, if possible, is often preferred to using references.

\section{Memory Allocation on Stack}
ATS supports memory being allocated in the stack frame of a function at
run-time. Like in the case of local variables, the type system of ATS
guarantees that no memory allocated in the stack frame of a function can be
accessed after the call to the function returns.

\begin{figure}[thp]
\input{DATS/alloca_example_1.dats}
\caption{An example involving memory allocation in stack frame at run-time}
\label{figure:alloca_example_1.dats}
\end{figure}
As an example, the code in Figure~\ref{figure:alloca_example_1.dats} prints
out the current time in certain string format. First, a call to the
function {\it time\_get} is issued to obtain the number of seconds since
the Epoch (the starting moment of the first of January, 1970 measured in
UTC time). Then a buffer of $N$ bytes is allocated in the stack frame of
the current function call, where $N$ equals some integer constant ${\it
CTIME\_BUFLEN}$ defined to be greater than or equal to $26$. The following
line in Figure~\ref{figure:alloca_example_1.dats} indicates a buffer of $N$
bytes is allocated in the current stack frame at run-time:
\begin{verbatim}
  var !p_buf with pf_buf = @[byte][CTIME_BUFLEN]()
\end{verbatim}
The dynamic variable ${\it p\_buf}$ is assigned the type $\tptr({\it
p\_buf})$, where ${\it p\_buf}$ is overloaded to refer to the starting
address of the allocated buffer, and the proof variable ${\it pf\_buf}$ is
given the view $@[{\it byte}?][N]@{\it p\_buf}$, which states that the
allocated buffer is uninitialized.  In general, given a type $T$, an
integer $I$ and an address $L$, the type $@[T][I]$ is for an array of $I$
values of type $T$ and the view $@[T][I]@L$ indicates that such an array is
stored at $L$.  If the allocated buffer needs to be initialized with some
byte value $b$, the following line can be used:
\begin{verbatim}
  var !p_buf with pf_buf = @[byte][CTIME_BUFLEN](b)
\end{verbatim}
In this case, the view assigned to ${\it pf\_buf}$ is $@[{\it
byte}][N]@{\it p\_buf}$, meaning that the buffer is initialized. The
function ${\it ctime\_r}$, which is a reentrant version of the function
${\it ctime}$, turns a time represented as the number of seconds since the
Epoch into some string representation and then stores the string inside the
buffer to which its last argumnent points. After the call to ${\it
ctime\_r}$ returns, the view of ${\it pf\_buf}$ changes into ${\it
strbuf}(N, I)@{\it p\_buf}$ for some natural number $I$, meaning that a
string (i.e., a sequence of bytes ending with the null byte) of length $I$
is stored in the buffer. After this string is printed out, the view of
${\it pf\_buf}$ needs to be changed into $@[{\it byte}?][N]@{\it p\_buf}$,
and this is done by calling the proof function ${\it
bytes\_v\_of\_strbuf\_v}$.

\section{Call-By-Reference}
The feature of call-by-reference in ATS is similar to the corresponding one
in \cplusplus. What is special in ATS is the way in which this feature is
handled by the type system.  In general, if $f$ is given a type of the
following form for some viewtypes $\VT_1$ and $\VT_2$:
%%
$$(\ldots,\&\VT_1\gg \VT_2,\ldots)\timp\ldots$$
%%
then a function call $f(\ldots,{\it x},\ldots)$ on some variable ${\it x}$
of the viewtype $\VT_1$ is to change the viewtype of ${\it x}$ into $\VT_2$
upon its return.  In the case where $\VT_1$ and $\VT_2$ are the same,
$\&\VT_1\gg \VT_2$ can simply be written as $\&\VT_1$. The variable $x$ may
be replaced with other forms of left-values.

\begin{figure}[thp]
\input{DATS/fact_function_cbr.dats}
\caption{An implementation of the factorial function that makes use of call-by-reference}
\label{figure:fact_function_cbr.dats}
\end{figure}
As an example, an implementation of the factorial function is given in
Figure~\ref{figure:fact_function_cbr.dats} that makes use of
call-by-reference. Note that if the line for introducing the variable
${\it res}$ in the implementation is replaced with the following one:
\begin{verbatim}
  val res: int = 1 // [res] is now a value, not a variable!
\end{verbatim}
then a type error should occurs as ${\it res}$ is no longer a left-value
when it is passed as an argument to ${\it loop}$. For instance, the reason
for introducing {\it ntick} as a variable in
Figure~\ref{figure:alloca_example_1.dats} is precisely due to ${\it
ctime\_r}$ requiring that its first non-proof argument be passed as a
reference.

The implementation in Figure~\ref{figure:fact_function_cbr.dats} should be
compared with the one in Figure~\ref{figure:fact_function_var.dats}. These
two are really the same implementation, but the latter is clearly cleaner
than the former in terms of the syntax being used.

\section{Dataviews}
A linear dataprop (for classifying linear proofs) is referred to as a {\it
dataview} in ATS.

\begin{figure}[thp]
\begin{verbatim}
dataview array_v (a:viewt@ype+, int(*size*), addr(*beg*)) =
  | {n:nat} {l:addr}
    array_v_cons (a, n+1, l) of (a @ l, array_v (a, n, l+sizeof a))
  | {l:addr} array_v_nil (a, 0, l)
\end{verbatim}
\caption{A dataview for modeling arrays}
\label{figure:dataview_array_v.dats}
\end{figure}
In Figure~\ref{figure:dataview_array_v.dats}, the dataview declaration
introduces a view constructor {\it array\_v} and two proof constructors
${\it array\_v\_nil}$ and ${\it array\_v\_cons}$ that are associated with
${\it array\_v}$.  Note that the sort $\sviewtaype$ is for classifying
viewtypes, that is, linear types of unknown size, and we use $\VT$ to range
over viewtypes (which include all types).  The types (or more precisely, props)
assigned to these two proof constructors are given as follows:
\[\begin{array}{rcl}
{\it array\_v\_nil} & : &
\forall a:\sviewtaype.\forall l:\saddr. () \timp {\it array\_v} (a, 0, l) \\
{\it array\_v\_cons} & : &
\forall a:\sviewtaype.\forall n:\snat.\forall l:\saddr. \\
&&\kern12pt
(a @ l, {\it array\_v} (a, n, l+{\it sizeof}(a))) \timp {\it array\_v} (a, n+1, l) \\
\end{array}\]
Given a viewtype $\VT$ (of unknown size), an integer $I$ and an address
$L$, ${\it array\_v}(\VT, I, L)$ is a view stating that there are $I$
values of the viewtype $\VT$ stored (in a row) at the memory location
$L$. A view as such is referred to as an array view.

\begin{figure}
\input{DATS/array_v_split_unsplit.dats}
\caption{Two proof functions for manipulating array views}
\label{figure:array_v_split_unsplit.dats}
\end{figure}
In Figure~\ref{figure:array_v_split_unsplit.dats}, two functions {\it
array\_v\_split} and {\it array\_v\_unsplit} are implmented for
manipulating array views. In essence, given a proof {\it pf} of the view
${\it array\_v} (\VT, N, L)$ for some viewtype $\VT$, integer $N$ and
address $L$, ${\it array\_v\_split}$ can be called to split {\it pf} (by
consuming it) into a pair of proofs ${\it pf}_1$ and ${\it pf}_2$ of the
views ${\it array\_v} (\VT, I, L)$ and ${\it array\_v} (\VT, N-I, L+{\it
OFS})$, respectively, for any natural number $I\leq N$, where ${\it OFS}$ equals
${\it sizeof}(\VT)$ multiplied by $I$. On the other hand, ${\it
array\_v\_unsplit}$ can be called to combine the views of two adjacently
allocated arrays into the view of a single array.  Note that the static
expression ${\it sizeof}(\VT)$ refers to the size of a viewtype $\VT$.

\begin{figure}
\input{DATS/array_get_set_elt_at.dats}
\caption{Two function templates for reading from and writing to a given array cell}
\label{figure:array_get_set_elt_at.dats}
\end{figure}
In Figure~\ref{figure:array_get_set_elt_at.dats}, the proof functions {\it
array\_v\_split} and {\it array\_v\_unsplit} are used in the implementation
of two function templates for reading from and writing to a given array
cell. Note that the dynamic expression ${\it sizeof}\langle\VT\rangle$
represents the size of a viewtype $\VT$, and the symbol ${\it imul2}$,
which is already given the infix status, refers to a function assigned the
following type:
\[\begin{array}{c}
\forall m:\sint.\forall n:\sint.
(\tint(m), \tint(n))\timp \exists p:\sint.({\it MUL}(m, n, p) \mid \tint (p)) \\
\end{array}\]
Clearly, when applied to two integers $m$ and $n$, ${\it imul2}$ returns
some integer $p$ and a proof stating that $p$ is the product of $m$ and
$n$.

\begin{figure}
\input{DATS/array_v_takeout.dats}
\caption{Another proof function for manipulating array views}
\label{figure:array_v_takeout.dats}
\end{figure}
In Figure~\ref{figure:array_v_takeout.dats}, a proof function {\it
array\_v\_takeout} is implemented.
Given a viewtype $\VT$, an address $L$, an integer $I$ and another integer
${\it OFS}$, a proof of the following view:
\[\begin{array}{c}
\VT @ (L+{\it OFS}) \timp {\it array\_v} (\VT, I, L)
\end{array}\]
is a linear function that returns a proof of the view ${\it array\_v} (\VT,
I, L)$ when applied to a proof of the view $\VT@(L+{\it OFS})$. In other
words, this linear function represents an array in which one array cell is
missing.  Therefore, the proof function ${\it array\_v\_takeout}$ turns the
view for an array into two views: one for a cell in the array and the other
for the rest of array, that is, the array minus the cell.
The proof functions ${\it
mul\_add\_cons}$ and ${\it mul\_elim}$ in the implementation are assigned
the following types:
\[\begin{array}{rcl}
{\it mul\_add\_cons} & : &
\forall i:\sint.\forall m:\sint.\forall n:\sint.~{\it MUL}(m,n,p)\timp {\it MUL}(m+i,n,p+i*n) \\
{\it mul\_elim} & : &
\forall m:\sint.\forall n:\sint.\forall p:\sint.~{\it MUL} (m, n, p)\timp (p = m*n)\Band\tvoid \\
\end{array}\]
When ${\it mul\_add\_cons}$ is called, the static variable $i$ should be
instantiated with an integer constant. Similarly, when ${\it mul\_elim}$ is
called, either the static variable $m$ or the static variable $n$ should be
instantiated with a constant. These two functions are declared as axioms in
the following file:
\begin{center}
${\it \$ATSHOME/prelude/SATS/arith.sats}$
\end{center}
The function templates ${\it array\_get\_elt\_at}$ and ${\it
array\_set\_elt\_at}$ are given another implementation in
Figure~\ref{figure:array_v_takeout.dats}, which makes use of ${\it
array\_v\_takeout}$.

\section{Dataviewtypes}
A linear datatype (for classifying linear values) is referred to as a {\it
dataviewtype} in ATS. As an example, ${\it list\_vt}$ is declared as a
dataviewtype in the following declaration:
\begin{verbatim}
dataviewtype list_vt (a:viewt@ype+, int) =
  | list_vt_nil (a, 0) | {n:nat} list_vt_cons (a, n+1) of (a, list_vt (a, n))
\end{verbatim}
The two data constructors associated with ${\it list\_vt}$ are assgined the
following types:
\[\begin{array}{rcl}
{\it list\_vt\_nil} & : &
\forall a:\sviewtaype.~()\timp{\it list\_vt}(a, 0) \\
{\it list\_vt\_cons} & : &
\forall a:\sviewtaype.\forall n:\snat.~(a, {\it list\_vt}(a, n))\timp{\it list\_vt}(a, n+1) \\
\end{array}\]
Assume that $C$ is a constructor of arity $n$ that forms a value
$C(v_1,\ldots,v_n)$ of some dataviewtype $T$ when applied to values
$v_1,\ldots,v_n$ of types $T_1,\ldots,T_n$.

\begin{figure}
\input{DATS/list_vt_free.dats}
\caption{A function template for freeing a given linear list}
\label{figure:list_vt_free.dats}
\end{figure}
A pattern of the form $\sim\!\!C(x_1,\ldots, x_n)$ is referred to as a
destruction pattern.  If a (linear) value $v$ of the form
$C(v_1,\ldots,v_n)$ matches the pattern $\sim\!\!C(x_1,\ldots, x_n)$, then
each $x_i$ is bound to $v_i$ for $1\leq i\leq n$, and {\em the memory for
storing $v$ is freed}. A typical case of using destruction patterns is
shown in Figure~\ref{figure:list_vt_free.dats}, where a function template
is implemented for freeing a given linear list.

\begin{figure}
\input{DATS/list_vt_length.dats}
\caption{A function template for computing the length of a given linear list}
\label{figure:list_vt_length.dats}
\end{figure}
We encounter a different situation when implementing a function for
computing the length of a given linear list. Instead of freeing the given
linear list, we need to preserve it (for later use). If a (linear) value
$v$ of the form $C(v_1,\ldots,v_n)$ matches a pattern of the form
$C(!x_1,\ldots,!x_n)$, then the type of the left-value holding $v$ changes
into $C(L_1,\ldots,L_n)$ for some addresses $L_1,\ldots,L_n$, and for each
$1\leq i\leq n$, $x_i$ is assgined the type $\tptr(L_i)$ and $v_i$ is
stored at $L_i$. This is referred to as unfolding a linear value.  Given a
left-value of the type $C(L_1,\ldots,L_n)$, applying ${\it fold@}$ to this
left-value turns its type into $T$ if the value stored at $L_i$ is $T_i$ for
each $1\leq i\leq n$. This is referred to as folding a linear value.  In
Figure~\ref{figure:list_vt_length.dats}, a function template for computing
the length of a given linear list is implemented that involves
folding/unfolding linear values.

%%% end of \chapter{Programming with Linear Types}
