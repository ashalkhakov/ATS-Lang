\chapter{Programming with Theorem Proving}

The paradigm of programming with theorem proving is rich and broad, and it
is probably the most innovative feature in ATS. It will become clear later
that this feature plays an indispensable role in ATS to support safe
manipulation of resources. In this chapter, we mainly give an introduction
to programming with theorem proving by presenting a few examples,
explaining some motivations behind this programming paradigm as well as
demonstrating a means to achieve it in ATS.

\begin{figure}[thp]
\input{DATS/list_concat_function.dats}
\caption{An implementation of list concatenation that does not typecheck}
\label{figure:list_concat_function}
\end{figure}
\section{Nonlinear Constraint Avoidance}
A function template {\it concat} is implemented in
Figure~\ref{figure:list_concat_function}, Given a list ${\it xss}$ of length
$m$ in which each element is of type $\tlist(T,n)$, ${\it
concat}\langle{T}\rangle({\it xss})$ constructs a list of type
$\tlist(T,m*n)$. When the first pattern matching clause in the code for
${\it concat}$ is typechecked, a constraint is generated that is
essentially like the following one:
\[\begin{array}{l}
\forall{m:\snat}.\forall{m_1:\snat}.\forall{n:\snat}. m=m_1+1\limplies n+(m_1*n) = m*n \\
\end{array}\]
This contraint may look simple, but it is rejected by the ATS constraint
solver as it contains nonlinear terms (e.g., $m_1*n$ and $m*n$).
In order to overcome the limitation, theorem-proving can be employed.

\begin{figure}[thp]
\begin{verbatim}
dataprop MUL (int, int, int) =
  | {n:int} MULbas (0, n, 0)
  | {m,n,p:int | m >= 0} MULind (m+1, n, p+n) of MUL (m, n, p)
  | {m,n,p:int | m > 0} MULneg (~m, n, ~p) of MUL (m, n, p)
\end{verbatim}
\caption{A dataprop for encoding integer multiplication}
\label{figure:dataprop_for_integer_multiplication}
\end{figure}
A dataprop declaration is given
in~\ref{figure:dataprop_for_integer_multiplication}. A dataprop is like a
datatype, but it can only be assigned to proof values (or proofs for
short).  In ATS, after a program passes typechecking, a procedure called
{\em proof erasure} can be performed to erase all the parts in the program
that are related to proofs. In particular, there is no proof construction
at run-time. The constructors ${\it MULbas}$, ${\it MULind}$ and ${\it
MULneg}$ associated with ${\it MUL}$ essentially correspond to the
following equations in a definition of integer multiplication (based on
integer addition):
\[\begin{array}{lll}
0 * n = 0 ~~~&
(m+1) * n = m * n + n~~\mbox{for $m>=0$}~~~&
(-m) * n = -(m * n)~~\mbox{for $m>0$} \\\
\end{array}\]
Given integers $m,n,p$, if ${\it MUL}(m,n,p)$ is inhabited, that is,
if it can be assigned to some proof value, then $m*n$ equals $p$.

\begin{figure}[thp]
\input{DATS/list_concat_function_proof.dats}
\caption{An implementation of list concatenation that does typecheck}
\label{figure:list_concat_function_proof}
\end{figure}
In Figure~\ref{figure:list_concat_function_proof}, another implementation
of the function template {\it concat} is given that avoids the generation
of nonlinear constraints. Given a type $T$, ${\it concat}\langle{T}\rangle$
is assigned the following type in this implementation:
\[\begin{array}{l}
\forall{m:\snat}.\forall{n:\snat}.~
\tlist(\tlist(T,n),m)\timp\exists{p:\snat}.~({\it MUL}(m,n,p)\mid\tlist(T,p)) \\
\end{array}\]
Given a list ${\it xss}$ of type $\tlist(\tlist(T,n),m)$, ${\it concat}(v)$
returns a pair $({\it pf}\mid{\it res})$ such that ${\it pf}$ is a proof of
type ${\it MUL}(m,n,p)$ for some $p$ and ${\it res}$ is a list of type
$\tlist(T,p)$. In other words, ${\it pf}$ acts a witness to $p=m*n$. After
proof erasure is performed, the implementation in
Figure~\ref{figure:list_concat_function_proof} is essentially translated
into the one in Figure~\ref{figure:list_concat_function}.

\begin{figure}
\input{DATS/factorial_function_proof.dats}
\caption{A fully verified implemenation of the factorial function}
\label{figure:factorial_function_proof}
\end{figure}
In Figure~\ref{figure:factorial_function_proof}, a dataprop
${\it FACT}$ is declared to encode the following definition of the factorial
function:
\[\begin{array}{ll}
fact (0) = 1 ~~~& fact (n+1) = fact (n) * (n+1)~~\mbox{for $n>0$} \\
\end{array}\]
Given integers $n$ and $r$, if ${\it FACT}(n,r)$ is inhabited, then
$fact(n)$ equals $r$.

\section{Proof Functions}
The following simple example depicts a typical scenario where
proof functions need to be constructed:
\begin{verbatim}
extern fun f {n:nat} (n: int n): bool
// the following function implementation does not typecheck
fun g {i:int} (i: int i) = f (i * i) // a nonlinear constraint is generated
\end{verbatim}
The function $f$ is assigned a type that indicates $f$ is from natural
numbers to booleans. Clearly, the constraint $\forall{i:\sint}.~i*i\geq 0$
is generated when the function $g$ is typechecked. This constraint is
rejected immediately as it is nonlinear. In order to avoid nonlinear
constraints, the following implementation of $g$ makes use of a proof
function ${\it lemma\_i\_mul\_i\_gte\_0}$:
\begin{verbatim}
extern prfun lemma_i_mul_i_gte_0
  {i,ii:int} (pf: MUL (i, i, ii)): [ii>=0] void
// end of [lemma_i_mul_i_gte_0]

fun g {i:int} (i: int i) = let
  val (pf | ii) = i imul2 i; prval () = lemma_i_mul_i_gte_0 (pf)
in
  f (ii)
end // end of [g]
\end{verbatim}
The type assigned to ${\it lemma\_i\_mul\_i\_gte\_0}$ indicates that ${\it
lemma\_i\_mul\_i\_gte\_0}$ proves $i*i\geq 0$ for every integer $i$.

\begin{figure}[thp]
\input{DATS/lemma_i_mul_i_gte_0_function.dats}
\caption{An implementation of a proof function showing $i*i\geq 0$ for every integer $i$}
\label{figure:lemma_i_mul_i_gte_0_function}
\end{figure}
An implementation of ${\it lemma\_i\_mul\_i\_gte\_0}$ is given in
Figure~\ref{figure:lemma_i_mul_i_gte_0_function}, where the auxiliary
proof functions ${\it aux1}$, ${\it aux2}$ and ${\it aux3}$ are given
the following types:
\[\begin{array}{rcl}
{\it aux1} & : &
\forall{m:\snat}.\forall{n:\sint}.\forall{p:\sint}.~{\it MUL}(m,n,p)\timp{\it MUL(m,n-1,p-m)} \\
{\it aux2} & : &
\forall{m:\snat}.\forall{n:\sint}.\forall{p:\sint}.~{\it MUL}(m,n,p)\timp{\it MUL(m,-n,-p)} \\
{\it aux3} & : &
\forall{n:\snat}.\forall{p:\sint}.~{\it MUL}(n,n,p)\timp (p\geq0)\Band\tvoid \\
\end{array}\]
In other words, the following is established by these proof functions:
\[\begin{array}{rcl}
{\it aux1} & {\rm proves} & \forall{m:\snat}.\forall{n:\sint}.~m*(n-1)=m*n - m \\
{\it aux2} & {\rm proves} & \forall{m:\snat}.\forall{n:\sint}.~m*(-n)=-(m*n) \\
{\it aux3} & {\rm proves} & \forall{n:\snat}.n*n\geq 0 \\
\end{array}\]

\begin{figure}
%%%\begin{verbatim}
\input{DATS/lemma_for_matrix_subscripting.dats}
%%%\end{verbatim}
\caption{A proof function needed in the implementation of matrix subscripting}
\label{figure:lemma_for_matrix_subscripting}
\end{figure}
\noindent{\bf Matrix Implementation}\kern6pt
A realistic example involving proof construction can be found in
the following file:
\begin{center}
{\it \$ATSHOME/prelude/DATS/matrix.dats}
\end{center}
where matrices are implemented in ATS. A 2-dimension matrix of dimension
$m\times n$ in ATS is represented a 1-dimension array of size $m\cdot n$ in
the row-major format. Given natural numbers $i$ and $j$ satisfying $i<m$
and $j<n$, the element in the matrix indexed by $(i,j)$ is the element in
the array indexed by $i\cdot n+j$. This means that the following theorem is
needed in order to implement matrix subscripting (without resorting to
run-time array bounds checking):
\[\begin{array}{l}
\forall m:\snat.\forall n:\snat.\forall i:\snat.\forall j:\snat.~
(i<m\land j<n)\limplies (i\cdot n + j < m\cdot n)
\end{array}\]
As linear constraints are handled automatically in ATS,
this theorem is equivalent to the following one,
\[\begin{array}{l}
\forall m:\snat.\forall n:\snat.\forall i:\snat.~
(i<m)\limplies (i\cdot n + n \leq m\cdot n)
\end{array}\]
which is encoded and proven in Figure~\ref{figure:lemma_for_matrix_subscripting}.

\begin{figure}[thp]
\input{DATS/intlst_datasort_example.dats}
\caption{A simple example involving datasort declaration}
\label{figure:intlst_datasort_example}
\end{figure}
\section{Datasorts}
So far, the type indexes appearing in the presented examples are all of
some built-in sorts (e.g., $\sbool$, $\sint$). In ATS, it is also possible
for the programmer to introduce sorts by datasort declaration. As an
example, a datasort {\it intlst} is first declared in
Figure~\ref{figure:intlst_datasort_example}, and each index of this sort
represents a sequence of integers. Subsequently, a dataprop {\it
int\_intlst\_lte} is declared, which captures the relation stating that a
given integer is less than or equal to each integer in a given integer
sequence.

The first proof function {\it int\_intlst\_lte\_lemma} in
Figure~\ref{figure:intlst_datasort_example} proves that an integer $i$ is
less than or equal to $x$ for each integer $x$ in a given integer sequence
if $i\leq j$ and $j\leq x$ for each integer $x$ in the sequence.

The next proof function {\it intlst\_lower\_bound\_lemma} in
Figure~\ref{figure:intlst_datasort_example} proves that for each integer
sequence, there is a lower bound $x_{lb}$ for the sequence, that is, $x\leq
x_{lb}$ holds for each $x$ in the sequence. Please notice the use of {\it
sif} and {\it scase} in this example. In constrast to {\it if}, {\it sif}
is used to construct a static conditional expression where the condition is
a static expression of the sort $\sbool$. In an analogous manner, ${\it
scase}$ is used to construct a static case-expression where the expression
being matched against is static. Each pattern used for matching in a static
case-expression must be simple, that is, it must be of the form of a
constructor being applied to variables.

%% \begin{verbatim}
%% dataparasort tm = lm of (tm -> tm) | app (tm, tm) // higher-order abstract syntax
%% \end{verbatim}

%%% end of \chapter{Programming with Theorem Proving} %%%
