\chapter{Introduction}
$\ATS$ is a typed programming language and the type system of $\ATS$ is an
{\em applied type system} rooted in the framework {\em Applied Type System}
($\AATS$), which also gives the programming language its name.

A primary motivation for developing $\AATS$ stems from an earlier attempt
to make (a form of) dependent types available for supporting practical
programming. While there is already a framework Pure Type System
($\PTS$)~\cite{barendregt92} that offers a simple and general approach to
designing and formalizing type systems, it is well understood that there
often exist some acute problems (especially in the presence of dependent
types) making it difficult for $\PTS$ to accommodate many common realistic
programming features.  In particular, we have learned that some great
efforts are required in order to maintain a style of pure reasoning as is
advocated in $\PTS$ when programming features such as general
recursion~\cite{constable87}, recursive types~\cite{mendler87},
effects~\cite{HMST95}, exceptions~\cite{HN88} and input/output are present.

To address such limitations of $\PTS$, $\AATS$ is developed to allow for
designing and formalizing type systems that can readily support common
realistic programming features.  The key salient feature of $\AATS$ lies in
a complete separation of statics, in which types are formed and
reasoned about, from dynamics, in which programs are constructed and
evaluated. This separation, with its origin in a previous study on a
restricted form of dependent types developed in Dependent ML
(DML)~\cite{XP99,XiThesis}, makes it feasible to support dependent types in
the presence of effects such as references and exceptions. Also, with the
introduction of two new (and thus unfamiliar) forms of types: {\em guarded
types} and {\em asserting types}, $\ATS$ is able to capture program
invariants in a more flexible and more effective manner than $\PTS$.

Currently, there are already a variety of programming paradigms that are
supported in $\ATS$ in a typeful manner, including:
\begin{itemize}
\item call-by-value functional programming, and
\item object-oriented programming with multiple inheritance, and
\item imperative programming with explicit pointers, and
\item meta-programming, and
\item modular programming.
\end{itemize}
Also, a form of (interactive) theorem proving is available in $\ATS$, with
which we can readily advocate a programming style that combines programming
with theorem proving. Furthermore, we may use $\ATS$ as a logical framework
to encode various deduction systems and their properties.

In the following chapters, we are to demonstrate that $\ATS$ is a rich and
unique language with a highly expressive type system that makes typeful
programming both {\em real} and {\em fun}.  We sincerely hope that you will
find it a joyful experience to program in $\ATS$, especially after seeing
some of wonders the type system of $\ATS$ can do for you. Furthermore, we
expect that before long you will also be able to share with other
programmers some of your joyful experience.

Now please buckle up and enjoy the ride!
